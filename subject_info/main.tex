\documentclass[12pt,a4paper]{article}
\usepackage[utf8]{inputenc}
\usepackage[margin=2.5cm]{geometry}
\usepackage{longtable}
\usepackage{array}
\usepackage{enumitem}
\usepackage{hyperref}
\usepackage{titlesec}
\usepackage{ulem}
\usepackage{xcolor}
\usepackage{fancyhdr}
\usepackage{graphicx}

\usepackage{tabularx}
\usepackage{pifont}
\newcommand{\checkbox}{\ding{113}} % empty checkbox




\titleformat{\section}{\bfseries\Large}{\thesection}{1em}{}
\titleformat{\subsection}{\bfseries\large}{\thesubsection}{1em}{}

\setlist[itemize]{left=0pt,label=--}
\setlength{\parindent}{0pt}
%\setlength{\parskip}{1em}

\begin{document}

\pagestyle{plain} % default for rest of document

\fancypagestyle{firstpage}{
  \fancyhf{} % clear default header/footer
  \fancyhead[L]{
    \includegraphics[height=1.2cm]{psychologo.png}\hspace{1em}
    \includegraphics[height=1.2cm]{mrilab.png}
  }
  \fancyhead[R]{
    \includegraphics[height=1.2cm]{unigrazlogo.png}
  }
  \renewcommand{\headrulewidth}{0pt} % no header line
}

\thispagestyle{firstpage}


\begin{center}
\LARGE\textbf{INFORMATION ÜBER EINE UNTERSUCHUNG IM MAGNETFELD FÜR PROBANDINNEN UND PROBANDEN}
\end{center}

\vspace{3em}

Die Magnetresonanztomographie (MRT) ist ein modernes, bildgebendes Verfahren, das die Anfertigung von Schnittbildern aller Körperregionen ermöglicht. Es kommen dabei \textbf{keine Röntgenstrahlen} oder \textbf{radioaktiven Substanzen} zum Einsatz. Die Bilder entstehen durch Signale von Wasserteilchen im Körper, die mittels eines starken Magnetfeldes und Hochfrequenzimpulsen erzeugt werden.

\vspace{1em}

\textbf{Nach aktuellem Wissensstand besteht bei den verwendeten Feldstärken kein Risiko einer biologischen Schädigung.}

\vspace{1em}

Für hochwertige Aufnahmen ist es wichtig, dass Sie während der Untersuchung \textbf{möglichst ruhig} im Gerät liegen. Bewegungen können zu Bildstörungen führen und eine Wiederholung der Aufnahmen notwendig machen.

Während der Untersuchung entsteht ein \textbf{lautes, klopfendes Geräusch}; Sie erhalten daher einen \textbf{Gehörschutz}. Außerdem bekommen Sie einen \textbf{Signalball}. Wenn Sie diesen drücken, wird die Untersuchung unterbrochen, und wir treten mit Ihnen in Kontakt.

Bitte bringen Sie \textbf{keine metallischen Gegenstände} (z.B. Schlüssel, Schmuck, Piercings, Haargummis, Brillen) oder \textbf{Kreditkarten} (werden gelöscht) in den Untersuchungsraum. 

Die Untersuchungsliege kann sich während der Messung bewegen. 

\textbf{Arme und Beine nicht überkreuzen!} Direkter Hautkontakt kann zu lokalen Erwär\-mungen führen. Bei Hitzegefühl bitte sofort den Signalball betätigen.

\vspace{1em}

\textbf{Das MRI Lab Graz führt ausschließlich Forschungsuntersuchungen durch.} Besteht ein Risiko (z.B. im Hinblick auf MR-Tauglichkeit oder Sicherheit), wird keine Untersuchung durchgeführt. Im Gegensatz zu klinischen Untersuchungen besteht \textbf{kein persönlicher Nutzen} für Sie.

\newpage
\section*{Persönliche Daten}

\noindent
\begin{tabular}{p{10cm} p{6cm}}  % Adjust widths as needed
  \begin{minipage}[t]{\linewidth}
    \vspace{0pt}  % aligns top
    \begin{tabular}{|p{4cm}|p{6cm}|}
      \hline
      \textbf{Name:} & \\
      \hline
      \textbf{Geburtsdatum:} & \\
      \hline
      \textbf{Körpergröße (cm):} & \\
      \hline
      \textbf{Gewicht (kg):} & \\
      \hline
    \end{tabular}
  \end{minipage}
  &
  \begin{minipage}[t]{\linewidth}
    \vspace{0pt}  % aligns top
    \hfill
    \fbox{%
      \begin{minipage}[c][2.5cm][c]{4cm}  % slightly taller to fit text below
        \centering
        \includegraphics[height=1.5cm]{../barcodes/barcode_<<ID>>.png}

        \vspace{0.2cm}

        \texttt{<<ID>>}
      \end{minipage}
    }
  \end{minipage}
\end{tabular}





\vspace{1.5em}

\section*{Fragebogen zur MR-Tauglichkeit}

Bitte beantworten Sie die folgenden Fragen sorgfältig:

\begin{longtable}{|p{13cm}|p{3cm}|}
\hline
Befinden sich metallische oder leitfähige Implantate oder Prothesen (z.B. Herzschrittmacher, Insulinpumpe, Mittelohrimplantate, Prothesen, Hörgeräte) in oder an Ihrem Körper? & Ja \checkbox\ Nein \checkbox \\
\hline
Haben Sie Metallteile (Piercing, Ohrringe, Zahnspangen etc.) oder Metallsplitter im Körper? & Ja \checkbox\ Nein \checkbox \\
\hline
Haben Sie Tätowierungen am Kopf-/Halsbereich, Nikotinpflaster oder kosmetische Augenmanipulationen? & Ja \checkbox\ Nein \checkbox \\
\hline
Wurde bei Ihnen eine Operation am Herz oder Kopf durchgeführt? & Ja \checkbox\ Nein \checkbox \\
\hline
Bestehen chronische Erkrankungen? \newline Wenn ja, welche: \rule{10cm}{0.4pt} & Ja \checkbox\ Nein \checkbox \\
\hline
Nehmen Sie derzeit Medikamente? \newline Wenn ja, welche: \rule{10cm}{0.4pt} & Ja \checkbox\ Nein \checkbox \\
\hline
\end{longtable}


\vspace{1.5em}

\section*{Zusätzliche Angaben für Frauen}

\begin{longtable}{|p{13cm}|p{3cm}|}
\hline
Besteht eine Schwangerschaft bzw. die Möglichkeit einer Schwangerschaft? &  Ja \checkbox\ Nein \checkbox \\
\hline
Verhüten Sie mit einer Kupfer- oder Goldspirale? &  Ja \checkbox\ Nein \checkbox \\
\hline
\end{longtable}

\vspace{1.5em}

\section*{Einverständniserklärung}

Ich bestätige, dass ich den Text gelesen, verstanden und die mich betreffenden Fragen nach bestem Wissen beantwortet habe. Ich stimme der Durchführung der MR-Forschungsuntersuchung zu. In einem persönlichen Gespräch wurden meine Fragen ausreichend beantwortet.

\vspace{1.5em}
\noindent

\textbf{Datum und Unterschrift:} \raisebox{-0.7ex}{\rule{10cm}{0.4pt}}


\vspace{2em}
\newpage
\section*{Datenschutzerklärung}

Die Universität Graz behandelt Ihre personenbezogenen Daten vertraulich gemäß den gesetzlichen Bestimmungen. Diese Erklärung informiert Sie gem. Art. 12, 13 DS-GVO über Zweck, Rechtsgrundlage und Ihre Rechte.

\textbf{Zweck und Rechtsgrundlage:} Die Daten \uline{Name, Geburtsdatum, Größe, Gewicht, Angaben zur MR-Tauglichkeit, MRT-Aufnahmen} werden auf Basis Ihrer Einwilligung verarbeitet. Die Speicherung erfolgt bis zum Widerruf, Dokumente zur Einwilligung max. drei Jahre ab Widerruf.

MR-Daten (DICOM) werden pseudonymisiert gespeichert. Bilddaten werden mit einem Defacing-Tool anonymisiert.

\textbf{Keine Weitergabe an externe Stellen.}

\textbf{Ihre Rechte:}
\begin{itemize}
\item Auskunft (Art. 15 DS-GVO)
\item Berichtigung, Löschung, Einschränkung (Art. 16–18 DS-GVO)
\item Datenübertragbarkeit (Art. 20 DS-GVO)
\item Widerspruch (Art. 21 DS-GVO)
\item Widerruf der Einwilligung (Art. 7 Abs. 3 DS-GVO)
\end{itemize}

\textbf{Beschwerderecht:} Österreichische Datenschutzbehörde, Wickenburggasse 8, 1080 Wien, E-Mail: \href{mailto:dsb@dsb.gv.at}{dsb@dsb.gv.at}

\textbf{Kontakt:}

Universität Graz, Institut für Psychologie/fMRT Taskforce, 8010 Graz \\
E-Mail: \href{mailto:psy.sek@uni-graz.at}{psy.sek@uni-graz.at} \\
Datenschutzbeauftragte: \href{mailto:dsba@uni-graz.at}{dsba@uni-graz.at} \\
Allgemeine Anfragen: \href{mailto:datenschutz@uni-graz.at}{datenschutz@uni-graz.at}

\vspace{2em}

\section*{Datenschutzrechtliche Einwilligung}

Ich willige ausdrücklich ein, dass die Universität Graz meine personenbezogenen Daten (\uline{Name, Geburtsdatum, Größe, Gewicht, Angaben zur MR-Tauglichkeit, MRT-Aufnahmen}) im Rahmen der MRT-Forschungsuntersuchung verarbeitet. 

Weiterverarbeitet werden das Alter in Jahren, die MRT-Aufnahmen (nach Defacing), sowie auswerterelevante MR-Parameter. Diese Einwilligung kann jederzeit per E-Mail an \href{mailto:psy.datenschutz@uni-graz.at}{psy.datenschutz@uni-graz.at} widerrufen werden.

Die Rechtmäßigkeit der bis zum Widerruf erfolgten Verarbeitung bleibt unberührt.

\vspace{1.5em}
\noindent

\textbf{Datum und Unterschrift:} \raisebox{-0.7ex}{\rule{10cm}{0.4pt}}



\end{document}
